\documentclass[a4]{article}
\usepackage{amsthm}
\usepackage{csquotes}
\usepackage{pifont}
\usepackage{booktabs}
\newcommand{\cmark}{\ding{51}}
\newcommand{\xmark}{\ding{55}}
\newtheorem{definition}{Definition}
\title{Analysing the cEDH Decklist Database}
\begin{document}
\maketitle
\section{Preliminaries}
Magic: The Gathering (MTG) cards are physical entities with many attributes, such as price, condition, weight, edition and so on.
In the context of this analysis though, we define cards and other terms as singular mathematical constructs with only the attributes necessary for the analysis.

\begin{definition}%{Color}
A \emph{color} is an element of the set of \emph{colors} $R = \{\textnormal{W},\textnormal{U},\textnormal{B},\textnormal{R}, \textnormal{G}\}$.
\end{definition}

\begin{definition}%{Card}
A \emph{card} is a triple $(s,O,I)$.
$s$ is its unique English card name, a string.
$O$, the colors and $I$, the color identity, are both subsets of $C$.
At the time of this writing there are 22570 cards, of which 1836 are used by at least one of the decks.
After applying a blacklist of very common cards such as basic lands and Talismans, the set of cards $C$ contains 1749 cards.
\end{definition}

\begin{definition}%{Deck}
In general, an MTG EDH \emph{deck} contains exactly 100 cards, of which basic lands and \enquote{Relentless Rats} may occur multiple times, but due to the blacklist,
 we define a \emph{deck} as a set (unordered collection without duplicates) of at most 100 cards.
\end{definition}

\begin{definition}%cEDH Decklist Database
The Competitive EDH Decklist Database, in the following \emph{database}, is a set of decks.
At this time there are 267 decklist links, 250 of which could be succesfully crawled and form our set $D$ of decklists.
\end{definition}

\begin{definition}{One-Hot encoding}
To allow mathematical operations on decks and cards, we transform them into vectors using \emph{one-hot encoding}:
We apply an arbitrary but consistent ordering on the set of cards $C$.
Each deck $d$ is represented as a $|C|$-dimensional vector that contains value 1 at each index in that ordering where the deck contains that card, and a 0 otherwise.
Similarily, but perhaps less intuitively, we apply such an ordering on $D$ and we represent each card $c$ as a $D$-dimensional vector that signifies which decks contain that card.
\end{definition}

\paragraph{Example}
For brevity, assume the set of cards are just\\
$C = \{\textnormal{Lightning Bolt}, \textnormal{Mystic Remora}, \textnormal{Rhystic Study}, \textnormal{Swords to Plowshares}, \textnormal{Necropotence}\}$.\\
And our decks are:\\
$D_1 = \{\textnormal{Lightning Bolt}, \textnormal{Necropotence}\}$.\\
$D_2 = \{\textnormal{Lightning Bolt}, \textnormal{Mystic Remora}, \textnormal{Rhystic Study}\}$.\\
The one-hot encodings of those decks are then $d_1 = (1,0,0,0,1) $ and $d_2 = (1,1,1,0,0)$
The one-hot encodings of those cards are:\\
\begin{tabular}{llll}
\toprule
			&in deck 1	&in deck 2	&encoding\\
\midrule
Lightning Bolt		&\cmark		&\cmark		&(1,1)\\
Mystic Remora		&\xmark		&\cmark		&(0,1)\\
Rhystic Study		&\xmark		&\cmark		&(0,1)\\
Swords to Plowshares	&\xmark		&\xmark		&(0,0)\\
Necropotence		&\cmark		&\xmark		&(1,0)\\
\bottomrule
\end{tabular}

\begin{definition}{Distance}
To compare, how similar or dissimilar any two decks or cards are, we need a distance function that takes two vectors as input and returns a numerical value that is high when the decks are dissimilar, and low when they are similar to each other.
On the flipside, a similarity function will return a high value, when the vectors are similar and a low (or negative) value, when they are very different.
\end{definition}

\begin{definition}{Euclidean Distance}
A commonly used distance function between vectors that the reader probably knows from school is the euclidean distance following the Pytagorean theorem defined as $\sqrt{(a-b)^2}$.
In our example, the distance between Lightning Bolt and Mystic Remora is 1, and the distance between Lightning Bolt and Swords to Plowshares is $\sqrt{2} \approx 1.414$.
The distance between Mystic Remora and Rhystic Study is 0, as they are mathematically identical in this representation, even though the actual playing cards are different.
One of the goals of this analysis is finding such pairwise similar cards.
For example, to recommend users of an application to add Mystic Remora to their deck if it already contains Rhystic Study.
Or to take a deeper look into decks that contain one of them but not the other if there is either a deckbuilding error or a deeper insight to be found.
Unfortunately... curse of dimensionality ... fractions ...
add reference to curse of dimensionality
\end{definition}

\begin{definition}{Manhattan-Distance (L1)}
L1 to the rescue..
\end{definition}

\section{Challenges}
\subsection{Lack of APIs}
The cEDH Decklist Database is a website without a public API or a publicly available dump in a structured format.
However the maintainers are very 
It contains links to decks mostly \emph{Moxfield}, which makes heavy use of JavaScript to populate its content and refuses to publish an API despite multiple requests by different people.
This makes bulk access time consuming and error prone, as a Selenium based web scrapper was cancelled after not finishing in 3 hours and using more than 40 GB of RAM using an Intel i9-10900k CPU.\footnotemark{}
\footnotetext{It is possible that the resource usage wasn't originally that extreme but was caused by a change in the website, which caused excessive retries, which stresses the importance of an API.}
22570 loaded cards
252
\subsection{Resource usage}
To prevent unnecessary traffic and thus service slowdown or increased costs for the maintainers of the APIs or crawled websites, intermediate results are cached.
If you use Docker or other methods that may not preserve those cache files, please keep this in mind.
If you reproduce the results obtained here those caches may cause outdated results.
In this case, try deleting the cache files `decks.json` and `cards.json` (Scryfall).

\section{Acknowledgements}
I am very thankful for the help in the database discord, but as the individuals seem to be privacy concious, I will only use their aliases and not link to their repositories.
\enquote{TWICE} for providing me with the Moxfield web scrapper
\enquote{Gerbrand}
\end{document}
